% -------------------------
% ResumeTex
% Resume/CV template
% Supragya Raj
% License: MIT License
% -------------------------

\documentclass[a4paper,6pt]{article}
\usepackage{tabularx}
\usepackage[left=0.5cm,right=1cm,bottom=1.5cm,top=1.5cm,bindingoffset=0.5cm]{geometry}
\usepackage{titlesec}
\usepackage{enumitem}
\usepackage{hyperref}
\usepackage{graphicx}
\graphicspath{ {./images/} }
\usepackage{xcolor}
\definecolor{mygray}{gray}{0.4}
\usepackage[sfdefault,light]{FiraSans} %% option 'sfdefault' activates Fira Sans as the default text font
\usepackage[T1]{fontenc}
\renewcommand*\oldstylenums[1]{{\firaoldstyle #1}}

% Margin adjustments
%\addtolength{\oddsidemargin}{-1.5in}
%\addtolength{\evensidemargin}{-1.5in}
% Sans serif fontface
% \renewcommand{\familydefault}{\sfdefault}

\titleformat{\section}
  {\normalfont\scshape}{\thesection}{1em}{}

\setlength{\parindent}{0px}
\addtolength{\textheight}{1.0in}
\linespread{0}
\setlength{\parskip}{0.3ex}

\setlist{nolistsep}

\begin{document}
\begin{center}
\Large \textbf{SUPRAGYA RAJ}


\end{center}
\begin{tabularx}{\textwidth}{X r}
	\href{https://www.linkedin.com/in/supragyaraj}{https://www.linkedin.com/in/supragyaraj} & Email: \href{mailto:supragyaraj@gmail.comj}{supragyaraj@gmail.com} \\
	\href{http://github.com/supragya}{http://github.com/supragya} & +91 97907 22967 
\end{tabularx}


\section*{Experience}
\vspace{-8px}
\hrule


\vspace{4px}
\hspace{5px}
\begin{tabularx}{\textwidth}{X r}
	\large{\textbf{Browserstack }} \small Mumbai, MH, India& \\
	\textit{\small Software Engineer (Desktop Platform team)} & \textit{Dec 2019 - Present} \\
\end{tabularx}

\small
\begin{itemize}
	\itemsep0em
	\item Actively Developing for and managing (Dev + Ops) a highly heterogeneous cloud infrastructure with over 7000 active terminals consisting of machines running all macintosh and windows platforms.
	\item Current cloud infrastructure consists of \textcolor{mygray}{\textbf{disparate components}} such as AWS, Windows KMS, VMware ESXi, reverse proxies, jumphosts, smokeping, DHCP etc.
    \item \textcolor{mygray}{\textbf{Hera SPOC framework}}: Architected and developed a single point of contact framework + tool for internal engineering teams to access, debug, automate and run diagnostics parallelly on terminals. 
    \item Hera allows status checks, configuration and actions to be taken from a \textcolor{mygray}{\textbf{single control panel}}.
    \item \textcolor{mygray}{\textbf{Parallel diagnostics}} allows diagnostics tools to be provided as an endpoint to different teams. Also, proves vital for actively monitoring production health and uptime.
  \end{itemize}

\vspace{4px}
\hspace{5px}
\begin{tabularx}{\textwidth}{X r}
	\large{\textbf{Cisco Systems}} \small Bangalore, KA, India& \\
	\textit{\small Software Engineer (Enterprise wireless controller team)} & \textit{July 2019 - Dec 2019} \\
	\textit{\small Software Engineering Intern  (Enterprise wireless controller team)}& \textit{Jan 2019 - June 2019} \\
\end{tabularx}

\small
\begin{itemize}
	\itemsep0em
	\item \textcolor{mygray}{\textbf{Fast wireless swarm upgrades}}: Designed and implemented heirarchial (pre)download mechanism allowing enterprise grade access points to download device images in a peer to peer fashion which earlier used to be central download based architecture. 
        \item The Fast wireless swarm upgrade mechanism reduced bandwidth load on Cisco wireless controllers and sped up download times from \textcolor{mygray}{\textbf{O(n)}} time to \textcolor{mygray}{\textbf{O(logn)}}, becoming highly effective in enterprise deployments as well as deployments with a remotely connected controller.
        \item The Fast wireless swarm upgrades system allowed a speed up of about \textcolor{mygray}{\textbf{6 times}}, reducing average swarm upgrade times from 90 minutes to 15-20 minutes per image upgrade.
	\item  Implementing controller side support for \textcolor{mygray}{\textbf{802.11r (wireless fast transition)}}, allowing fast roaming between Cisco enterprise grade WiFi access points, lowering roam times by around \textcolor{mygray}{\textbf{10 times}}.
    \item \textcolor{mygray}{\textbf{Configuration translator}}: Developed mapping layer code to translate \textcolor{mygray}{\textbf{openconfig}} standard based wireless management to commands usable by Cisco devices. 
\end{itemize}

\section*{Projects and Open source contributions}
\vspace{-8px}
\hrule
\vspace{4px}


\hspace{5px}
\begin{tabularx}{\textwidth}{X r}
	{\textbf{\href{https://github.com/supragya/HLang}{HLang Shell Language and Interpreter}}} \small & \\
\end{tabularx}
\begin{itemize}
    \item \small Developed a scripting language and its interpreter to provide a subset of functionalities provided by the Bourne Shell (bash) on an opensource \textcolor{mygray}{\textbf{microkernel}} (HelenOS) operating system.
    \item \small The system included building AST, compile time optimisations etc. 
\normalsize
\end{itemize}



\hspace{5px}
\begin{tabularx}{\textwidth}{X r}
	{\textbf{\href{https://github.com/supragya/AXIOM_RawStreamHandler}{Raw video container format}}} \small Google Summer of Code 2018& \\
\end{tabularx}
\begin{itemize}
    \item \small Extending an already existing video file format: Magic Lantern's MLV video container (which existed for Canon cameras) to be used by apertus open source cameras. 
    \item \small This allows video recording to be done straight onto a well supported video file format: MLV instead of only earlier option: image sequences.
    \item \small Implemented, tested and benchmarked the performance of our usage of MLV file format on \textcolor{mygray}{\textbf{RAID 0 configs}}.
\normalsize
\end{itemize}

\hspace{5px}
\begin{tabularx}{\textwidth}{X r}
	{\textbf{\href{https://github.com/supragya/libfuse-FrameServer}{libfuse-FrameServer}}} \small Google Summer of Code 2018, Mentor - 2019& \\
\end{tabularx}
\begin{itemize}
    \item \small A pseudo file system implementation based on libfuse to provide \textcolor{mygray}{\textbf{RGBA video output}} for applications such as VLC, Adobe Premiere Pro from a \textcolor{mygray}{\textbf{raw stream of camera sensor output voltages}}.
    \item \small The frameserver acts as a middleman which processes raw stream to RGBA values on the other end.
    \item \small The framesever allowed one to control elements of stream processing such as \textcolor{mygray}{\textbf{HW acceleration, denoising, demosaicing, downscaling etc}}.
    \item \small A related precursor project PiNG12RAW recieved \textcolor{mygray}{\textbf{over 120 forks on github}}. 
    \normalsize
\end{itemize}


\section*{Publications}
\vspace{-8px}
\hrule
\vspace{8px}
\begin{itemize}
	\item \small Raj S., Chodnekar S.P., Harish T., Sriraman H. (2019) \textcolor{mygray}{\textbf{eMDPM: Efficient Multidimensional Pattern Matching Algorithm for GPU}}. In: Tiwari S., Trivedi M., Mishra K., Misra A., Kumar K. (eds) Smart Innovations in Communication and Computational Sciences. Advances in Intelligent Systems and Computing, vol 851. Springer, Singapore. 
\normalsize
\end{itemize}


\section*{Education}
\vspace{-8px}
\hrule
\vspace{4px}
\hspace{5px}
\begin{tabularx}{\textwidth}{X r}
	\textbf{Vellore Institute of Technology} & Chennai, India \\
	\textit{\small Bachelor of Technology in Computer Science and Engineering \textcolor{mygray}{\textbf{GPA: 9.02/10.0}}} & \textit{July, 2019} 
\end{tabularx}

\vspace{5px}
\hspace{5px}
\begin{tabularx}{\textwidth}{X r}
	\textbf{Army Public School, Shankar Vihar} & New Delhi, India \\
	\textit{\small Senior Secondary, (PCM + CS), 12th CBSE:  \textcolor{mygray}{\textbf{95.6\% (aggregate), 98\% (CS)}}} & \textit{May 2015} \\
	\textit{\small High School, 10th CBSE, \textcolor{mygray}{\textbf{CGPA 10.0}}} & \textit{May 2013}
\end{tabularx}


\section*{Awards and achievements}
\vspace{-8px}
\hrule
\vspace{8px}
\begin{itemize}
	\item \small \textbf{Winner, Smart India Hackathon 2019, Goldman Sachs}: Mentored the team from IGDTUW, Delhi in GS problem set - smart transportation held at NIT, Calicut.
	\item \small \textbf{\href{https://github.com/IOStream-OpenEnd}{IOStream OpenEnd}}: \small A youtube based channel and open source initiative to mentor students for GSoC. The community worked on multiple projects ranging from Audio Fingerprinting to Automatic Trailer generation mechanisms.
\end{itemize}

\end{document}
